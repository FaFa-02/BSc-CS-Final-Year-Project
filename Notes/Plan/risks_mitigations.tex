\section{Risks and Mitigation}

In life every decision we make comes with risks and this research project is no different. During this project I will do my best to reduce these risks where I can. I also added some extension work to the plan which I would love to achieve but I recognise that this may not be entirely possible due to time constraints. In this section I will discuss some of the possible risks that I might encounter throughout the project and how I plan to mitigate them.

\subsection{Data Loss}

There is always the possibility of hardware failure such as hard drive corruption, causing the loss of saved files which can contain research and software implementations. This can happen at any time and there is little that can be done to stop it from occurring. This can be mitigated by using web based repositories such as Git Lab to store up-to-date versions of my project online so that they can be restored if lost. 

\subsection{Over Ambitious Plan}

There is also the risk of lack of time management or having a plan which is over ambitious. The plan presented here is a tough one which requires lots of dedication. However, I also have other modules within this course that require my attention, meaning that I might need to cut some of the extensions out to ensure a full project is delivered in the end.

\subsection{Ethical and Bias Concerns}

A possible risk in this project is creating a model that helps to perpetuate social biases unintentionally. This can happen due to these biases being present within the datasets themselves, meaning that they need to be checked through visualisation before being used. The model should also be regularly re-trained with new data which should contain less social biases.

\subsection{Data Quality and Quantity}

The dataset itself might also have issues such as missing or incorrect data, lots of outliers or simply there not being enough data points to sufficiently train the model. This can again be mitigated by visualisation of the data before using it to train the model.

\subsection{Hardware Limitations}

Machine Learning algorithms can be memory intensive and hard to run on the average computer, especially considering the amount of features that are present withing Boston housing problem datasets. If my own devices are unable to handle these operations then I might need to check with my professors to see if the university has better resources where I could run the models.

\subsection{Balance Between Theory and Code}

There is also the risk of becoming to focused on either the theoretical and implementation side of the project and neglecting the programming/software engineering side of it or vice versa. This can be mitigated by doing constant self checks on the progress of the project. Making sure that both sides are moving forward together.
